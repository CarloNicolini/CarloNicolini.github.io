Caro Diego,

Dopo la chiaccherata con te, tornato in Italia, ho ripensato al discorso
di come adattare la Surprise al configuration model, perchè se ne
valesse la pena avremmo una funzione costo molto potente.

Iniziamo con un po' di notazione: - \[G\] grafo undirected, unweighted,
no self-loops, no multiedges. - \[n\] numero di nodi del grafo. - \[m\]
numero di lati del grafo. - \[p\] numero di coppie di vertici del grafo
\[p=\binom{n}{2}\]. - \[m_c\] numero di lati nella comunità \[c\]. -
\[p_c\] numero di coppie di vertici nella comunità \[c\]. - \[m_\zeta\]
numero totale di lati intracluster, \[m_\zeta=\sum_c m_c\]. -
\[p_\zeta\] numero totale di coppie intracluster,
\[p_\zeta=\sum_c p_c\]. - \[p-p_\zeta\] numero totale di coppie
extracluster. - \[m-m_\zeta\] numero totale di lati extracluster.

Partiamo dalla definizione originale di Surprise che nella sua
formulazione originale è basata sulla distribuzione cumulativa inversa
ipergeometrica,

\[
S = -\log \sum_{i=m_\zeta}^m \left( \frac{\binom{p_\zeta}{i}\binom{p-p_\zeta}{m-i}}{\binom{p}{m}} \right)
\]

Il null model implicito su cui è basata la Surprise è il modello
\[G_{nm}\] (e non Erdos-Renyi \[G_{np}\]) perchè il numero di lati è
fissato a \[m\] (è una versione microcanonica dell'insieme \[G_{np}\] in
fondo). La cardinalità dell'insieme \[G_{nm}\] con esattamente \[n\]
nodi ed \[m\] lati viene rappresentata al denominatore della Surprise
dal fattore

\[\binom{p}{m}^{-1}\]

ovvero esso è la cardinalità dell'insieme di tutti i possibili grafi con
esattamente \[m\] lati e \[p=\binom{n}{2}\] coppie di lati, o se vuoi
vederla diversamente, \[m\] coppie di vertici connesse da un lato e
\[p\] coppie di vertici totali. A quanto ne so io comunque i due insiemi
\[G_{nm}\] (Gilbert) e \[G_{np}\] (Erdos-Renyi) sono circa simili per
\[n\] grande.

Ora per semplificarci la vita consideriamo solamente il termine
dominante di \[S\] tenendo solo il primo termine della sommatoria
\[i=m_\zeta\] ed indichiamo il risultato \[S_D\]. Otteniamo quindi:

\[
S_{D} \approx -\log \left( \frac{\binom{p_\zeta}{m_\zeta}\binom{p-p_\zeta}{m-m_\zeta}}{\binom{p}{m}} \right)
\]

e questo è anche quello che fa Traag nel suo paper. Quindi in questa
formulazione Surprise ti dice qual'è la probabilità che, pescando \[m\]
palle da un'urna contenente \[p\] palle, di cui \[p_\zeta\] bianche e
\[p-p_\zeta\] nere, tu abbia in mano \emph{esattamente} \[m_\zeta\]
palle bianche.

Questa descrizione dal punto di vista statistico è un Fisher test
esatto, che ti dice quanto confidentemente si può rigettare l'ipotesi
nulla che la densità intracluster density \[m_\zeta/p_\zeta\] sia uguale
alla densità del grafo \[m/p\]. Teniamolo in mente per dopo.

Notiamo ora che in questa formulazione noi stiamo considerando solamente
due tipi di popolazioni, cioè la popolazione delle coppie intracluster e
quella delle coppie extracluster di cui esse possono formare lati o non
formarli.

Questo fatto si riflette nella formulazione asintotica della Surprise,
la Asymptotical Surprise che fa uso della divergenza di Kullback-Leibler
binaria sulle frequenze relative osservata ed aspettata della
popolazione intracluster e della popolazione extracluster.

\[
S_A = m D_{KL}(q \| \left< q\right>) 
\]

dove in questo caso
\[D_{KL}(x \| y)=x \log(x/y) + (1-x) \log( (1-x)/(1-y) )\].

Parlando con te mi avevi fatto notare che questo ha anche un altro
parallelo in statistica, il likelihood-ratio. Spulciando qua e là ho
scoperto che la statistica nota come G-test prende esattamente la forma
che ci piace e cioè, date \[N\] osservazioni, di cui \[O_i\] è il numero
di conteggi osservati, \[E_i\] il numero di conteggi aspettati dati un
certo modello nullo, allora il valore \[G\] si scrive come:

\[
G=2 \sum_i O_i \log(O_i/E_i)
\]

e questa statistica \[G\] è intimamente collegata alla divergenza KL
perchè

\[G=2 \sum_i O_i \log(O_i/E_i) = 2N \sum_i o_i \log(o_i/e_i)\]

dove, le frequenze relative osservate ed aspettate sono \[o_i\] ed
\[e_i\]. Questo è interessante di per se e lo useremo come fatto utile
fra un po', ma torniamo un secondo alla Surprise.

Come detto prima, la Surprise ha due proprietà principali. La prima è
che la popolazione delle biglie nell'urna è di due tipi: intracluster ed
extracluster. Da questo deriva il fatto che si modella con una
distribuzione ipergeometrica, approssimata da binomiale, riconducibile a
KL binaria. Se si considera invece un estensione ipergeometrica
multivariata infatti, che qui chiamiamo \[S_M\] e definiamo:

\[
S_{M} = -\log \left( \prod_{rs} \frac{\binom{n_r n_s}{m_{rs}} } {\binom{p}{m}} \right)
\]

allora questo mette in evidenza che la definizione originale è una sorta
di marginalizzazione (forse non è il termine esatto) su questa versione
multivariata. Quest'ultima definizione di Surprise che è stata
introdotta da Traag nel lavoro sulla Asymptotical Surprise, ha anch'essa
una versione asintotica (indicata qui come \[S_{MA}\] che si scrive
come:

\[
S_{MA} = \sum_{rs} m_{rs} \log \left( \frac{m_{rs}}{n_r n_s} \right)
\]

Ora, questa funzione costo era stata già proposta da Newman in un lavoro
sui stochastic block models, ed era stato notato che tendeva a
raggruppare nella stessa comunità i nodi con degree alto, poichè come è
evidente non tiene in conto la degree sequence, infatti nel modello
\[G_{nm}\] non vi è assolutamente nessuna menzione esplicita dei degree,
la rete è considerata uniforme (in media).

Newman allora aveva sostituito a \[n_r\] ed \[n_s\] le somme dei degree
dei nodi nella comunità \[r\] ed \[s\], ottenendo quello che lui ha
chiamato degree-corrected stochastic block model.

\[
S_{DCSBM} = \sum_{rs} m_{rs} \log \left( \frac{m_{rs}}{K_r K_s} \right)
\]

Incorporando la degree-sequence, implicitamente in questa funzione
vengono assegnate probabilità più alte ai lati i cui endpoints hanno
degree alto e questo dà la possibilità di fittare reti con una maggiore
likelihood.

Ragionando su questo ultimo passaggio, il messaggio che contiene è che
qui si passa dal considerare le \textbf{coppie} di vertici a considerare
gli \textbf{stubs}

Infatti questo con un po' di intuizione corrisponderebbe ad un modello
ad urna senza rimpiazzo (o con rimpiazzo non è importante se la rete è
grande) con un totale di \[C^2/2\] possibili tipi di palle (il fattore 2
perchè consideriamo il numero di lati dal blocco \[r\] al blocco \[s\]
uguale al numero di lati dal blocco \[s\] al blocco \[r\]). Risulterebbe
una cosa del genere:

\[
S_{CM} = -\log \left( \prod_{rs}  \frac{\binom{K_r K_s}{m_{rs}}}{\textrm{const}} \right) \approx S_{DCSBM}\]

dove \[\textrm{const}\] è una costante di normalizzazione che dopo
averci ragionato per parallelismo con la definizione di Surprise data
prima, dovrebbe essere data dal numero di possibili grafi che possono
esistere nel configuration model con quella una sequenza di degree
\[\{k_1, \ldots ,k_n\}\]. Da argomenti combinatoriali risulta che la
cardinalità \[| \Omega_{CM}|\] del configuration model data tale degree
sequence si calcola con il coefficiente multinomiale:

\[
| \Omega_{CM} | = \binom{2m}{k_1, \ldots k_n} = \frac{(2m)!}{\prod \limits_i^n (k_i)!}
\]

Questo vorrebbe dire che riconsiderando tutto il ragionamento fatto
finora per la Surprise all'indietro, nonostante non ci sia una
dimostrazione passo passo (che non dovrebbe essere troppo difficile),
possiamo ottenere una forma di Surprise adattata al configuration model
che in definitiva sarebbe data da:

\[
S_{CM} = \prod_{rs} \frac{ \binom{K_r K_s}{m_{rs}}}{\binom{2m}{k_1, \ldots k_n}}
\]
